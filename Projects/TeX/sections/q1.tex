\section*{Question 1}

How motors and encoders interface with the Microbit board.

\subsection*{Solution}

The motors are controlled via PWM signals generated by the \texttt{nRF52833} microcontroller.
The speed and direction of the motors are adjusted by varying the duty cycle of these PWM signals, and by appropriately setting the direction pins of the motor driver, respectively.
The encoders, which are connected to the \texttt{GPIO} pins of the microcontroller, generate pulses as the motors rotate.
These pulses are counted and timed to calculate motor speed in \texttt{RPM} (revolutions per minute).
The system uses interrupts to handle pulse events efficiently and timers to calculate the time difference between pulses, ensuring accurate speed measurement in a non-blocking way.
This is done by configuring the \texttt{GPIOTE} registers to track the rising and falling edges of the encoder pulses.

The motors and encoders are interfaced with the \texttt{micro:bit} board using the GPIO (General Purpose Input/Output) pins of the microcontroller.
The \texttt{micro:bit} was connected to the edge connector and the power is supplied to the motor controller from a battery.
UART was used to get data from the \texttt{micro:bit} to print the encoder output.
The GPIO registers we of type \texttt{volatile long *} so that the compiler does not optimize access to register because this register may get modified due to the hardware or the interrupts.
The base address for the \texttt{GPIO0} and \texttt{GPIO1} are \hex{0x50000000UL} and \hex{0x50000300UL}, respectively.
According to whether the pin is \texttt{INPUT} or \texttt{OUTPUT}, the direction bit and the pin configuration is set using \texttt{pinMode\@()} function.

The header file \texttt{nrf52833.h} is used to get the definitions and functions specific to the \texttt{nRF52833} which the \texttt{micro:bit} contains.
The pulse modulation unit is taken from the \texttt{nrf52833.h} file.
The PWM clock frequency is set to 1MHz and the PWM frequency is set to 500 Hz, which is the frequency at which the PWM signal will oscillate.
The period of the PWM is calculated based on the clock frequency and desired PWM frequency, and is set in the \texttt{COUNTERTOP} register, thereby a PWM waveform is created whenever the counter value reaches 0.
The pins on which the motor is connected is set to \texttt{OUTPUT}.
A \texttt{motor\_init\@()} function is defined to initialize the motor.
Inside this the required pins are set to the PWM output and the PWM is enabled so that the pins generate PWM output.
A function \texttt{motor\_on\@()} is also provided so that PWM sequence is generated when we provide the pin, direction and duty cycle.
Similar to the motor, there are functions in encoders that are defined.
In the \texttt{encoder\_init\@()}, the pins to which the encoder is connected are set to \texttt{INPUT} and the digital interrupts are enabled so that we get the input from the encoder when the readings in the encoder changes.
