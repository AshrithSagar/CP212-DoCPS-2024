\section*{Question 3}

What are the sizes of various sections in your executable?\\
What is the total ROM and RAM requirement?

\subsection*{Solution}

\subsubsection*{Sizes of various sections}

We can observe the size of each of the section in the third column in Table~\ref{tab:map}.

\begin{table}[htbp]
    \centering
    \resizebox{0.9\textwidth}{!}{
        \begin{tabular}{|l|l|l|l|l|}
	\hline
	\textbf{Symbol}        & \textbf{Start Address} & \textbf{End Address} & \textbf{Type} & \textbf{Section}       \\ \hline
	.rodata.\_\_Vectors    & \hex{0x00000000}       & \hex{0x00000040}     & Data          & .rodata.\_\_Vectors    \\ \hline
	.text.Default\_Handler & \hex{0x00000194}       & \hex{0x00000198}     & Code          & .text.Default\_Handler \\ \hline
	.text.Reset\_Handler   & \hex{0x00000198}       & \hex{0x000001a2}     & Code          & .text.Reset\_Handler   \\ \hline
	.text.buttonInit       & \hex{0x000001a4}       & \hex{0x000001be}     & Code          & .text.buttonInit       \\ \hline
	.text.digitalRead      & \hex{0x000001c0}       & \hex{0x00000208}     & Code          & .text.digitalRead      \\ \hline
	.text.digitalWrite     & \hex{0x00000208}       & \hex{0x0000027c}     & Code          & .text.digitalWrite     \\ \hline
	.text.displayImage     & \hex{0x0000027c}       & \hex{0x0000033a}     & Code          & .text.displayImage     \\ \hline
	.text.displayInit      & \hex{0x0000033c}       & \hex{0x000003a6}     & Code          & .text.displayInit      \\ \hline
	.text.main             & \hex{0x000003a8}       & \hex{0x000003f8}     & Code          & .text.main             \\ \hline
	.text.naiveDelay       & \hex{0x000003f8}       & \hex{0x0000041c}     & Code          & .text.naiveDelay       \\ \hline
	.text.pinMode          & \hex{0x0000041c}       & \hex{0x000004ac}     & Code          & .text.pinMode          \\ \hline
	.rodata.LED\_COL\_PINS & \hex{0x000004c4}       & \hex{0x000004d8}     & Data          & .rodata.LED\_COL\_PINS \\ \hline
	.rodata.LED\_ROW\_PINS & \hex{0x000004d8}       & \hex{0x000004ec}     & Data          & .rodata.LED\_ROW\_PINS \\ \hline
	.rodata.pics           & \hex{0x000004ec}       & \hex{0x00000537}     & Data          & .rodata.pics           \\ \hline
	.bss                   & \hex{0x20000000}       & \hex{0x20000060}     & Zero          & .bss                   \\ \hline
	.bss.stack\_memory     & \hex{0x20000060}       & \hex{0x200000e0}     & Zero          & .bss.stack\_memory     \\ \hline
\end{tabular}

    }
    \caption{
        Memory Map
    }\label{tab:map}
\end{table}

\clearpage
\subsubsection*{Total ROM and RAM requirements}

When the program is compiled, we get the following log,
\[
    \texttt{
        Program Size: Code=104 RO-data=72 RW-data=32 ZI-data=144
    }
\]
where
\begin{itemize}[itemsep=0pt, topsep=0pt, partopsep=0pt, parsep=0pt]
    \item \( \texttt{RO} \): Read-Only data
    \item \( \texttt{RW} \): Read-Write data
    \item \( \texttt{ZI} \): Zero-Initialised data
\end{itemize}

\vspace*{1em}
In the program, the stack size was considered to be of 128 bytes.

The total ROM and RAM requirements for the program can be seen in the \( \texttt{.map} \) file that is generated after compilation.

\begin{table}[htbp]
    \centering
    \begin{tabular}{@{}ll@{}}
	\toprule
	\textbf{Description}                      & \textbf{Total Size} \\
	\midrule
	Total RO Size (Code + RO Data)            & 176 (0.17 kB)       \\
	Total RW Size (RW Data + ZI Data)         & 164 (0.16 kB)       \\
	Total ROM Size (Code + RO Data + RW Data) & 196 (0.19 kB)       \\
	\bottomrule
\end{tabular}

\end{table}

It can also be calculated from the sizes as given in the third column of Table~\ref{tab:map}.
\begin{align*}
    \text{Total RO size}
     & =
    \text{Code} + \text{RO Data}
    \\ & =
    (64 + 4 + 10 + 2 + 2 + 2 + 16 + 66 + 2 + 4 + 4)
    = 176 \text{ bytes}
    \\
    \text{Total RW size}
     & =
    \text{RW Data} + \text{ZI Data}
    \\ & =
    (4 + 16) + (128 + 16) = 164 \text{ bytes}
    \\
    \text{Total ROM size}
     & =
    \text{Code} + \text{RO Data} + \text{RW Data}
    \\ & =
    (176) + (4 + 16) = 196 \text{ bytes}
\end{align*}

\( \therefore \) The \underline{total ROM requirement is 196 bytes} while the \underline{total RAM requirement is 164 bytes}.
