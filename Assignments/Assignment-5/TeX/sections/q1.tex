\section*{Question 1}

View \(\texttt{m}\) and \(\texttt{c}\) in Memory Window by typing addresses as \(\texttt{\&m}\) and \(\texttt{\&c}\).\\
Are they initialised properly?
Why or why not?

\subsection*{Solution}

\begin{figure*}[htbp]
    \centering
    \includegraphics[width=0.8\textwidth]{figures/images/m.png}
    \caption{
        Memory Window for \texttt{\&m}.
        \( \quad \texttt{m} = \hex{0x00000007} \)
    }\label{fig:m}
\end{figure*}

\begin{figure*}[htbp]
    \centering
    \includegraphics[width=0.8\textwidth]{figures/images/c.png}
    \caption{
        Memory Window for \texttt{\&c}.
        \( \quad \texttt{c} = \hex{0x0000000B} \)
    }\label{fig:c}
\end{figure*}

The Memory Windows for \(\texttt{\&m}\) and \(\texttt{\&c}\) are shown in Figure~\ref{fig:m} and Figure~\ref{fig:c} respectively.

The values of \(\texttt{m}\) and \(\texttt{c}\) are initialised properly when we run the program with the \( \texttt{--no\_remove} \) option set.
If not, the default optimisations by the compiler would remove these constants and hardcode them in the instructions, since they are only used once.
Thereby, we wouldn't be able to see these values as shown in the Symbols window in Figure~\ref{fig:symbols}.

The values of \(\texttt{m}\) and \(\texttt{c}\) are stored in the read-only data section, which is copied to RAM during startup, since they are read-only initialised data.

Hence, \underline{both \(\texttt{m}\) and \(\texttt{c}\) are initialised properly} with the \( \texttt{--no\_remove} \) option enabled.
