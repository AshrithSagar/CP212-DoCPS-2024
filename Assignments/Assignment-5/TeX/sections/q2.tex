\section*{Question 2}

Next, view \(\texttt{x\_array}\) and \(\texttt{y\_array}\) in Memory Window.
What are their values?\\
Are they properly initialised?
Why or why not?

\subsection*{Solution}

\begin{figure*}[htbp]
    \centering
    \includegraphics[width=0.8\textwidth]{figures/images/x_array.png}
    \caption{
        Memory Window for \texttt{x\_array}.
    }\label{fig:x_array}
\end{figure*}

\begin{figure*}[htbp]
    \centering
    \includegraphics[width=0.8\textwidth]{figures/images/y_array.png}
    \caption{
        Memory Window for \texttt{y\_array}.
    }\label{fig:y_array}
\end{figure*}

\begin{figure*}[htbp]
    \centering
    \includegraphics[width=0.8\textwidth]{figures/images/initial.png}
    \caption{
        Watch window before running the program.
    }\label{fig:initial}
\end{figure*}

When the \( \texttt{\_\_main} \) function is overridden in \( \texttt{start.c} \), the C Runtime initialisation code is not executed, which is responsible for initialising the values from the ROM into the RAM.\@
Thereby, for \( \texttt{x\_array} \), which is supposed to be Read-Write initialised value, has it's values as some garbage values as shown in Figure~\ref{fig:x_array}.

For \( \texttt{y\_array} \), which is supposed to be Read-Write uninitialised value, has it's values as some garbage values as shown in Figure~\ref{fig:y_array}, which is expected.

The Watch Window before running the program is shown in Figure~\ref{fig:initial} where we can observe the garbage values that are stored in the two arrays.

If we had not overrided the original \( \texttt{\_\_main} \) function that performs the C Runtime initialisations, we would have observed the correct initialised values in the RAM.\@

Hence,
\begin{enumerate}[itemsep=0pt, topsep=0pt, partopsep=0pt, parsep=0pt]
    \item No, \underline{\( \texttt{x\_array} \) is not properly initialised}, when \( \texttt{\_\_main} \) function is overridden.
    \item The \underline{initialisation for \( \texttt{y\_array} \) doesn't matter} in this case, since it is anyway expected to be garbage values due to it being Read-Write uninitialised data.
\end{enumerate}
