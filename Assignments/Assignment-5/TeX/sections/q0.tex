\section*{Minimal system}

In this exercise, you will implement a standalone system with your own startup code.

\begin{enumerate}[itemsep=0pt]
    \item Create project without CMSIS Core and Device.
    \item Create a \(\texttt{start.c}\) file as discussed in the lab session.
    \item Add two files supplied with this assignment:
          \begin{itemize}
              \item \(\texttt{linear.c}\): Implements integer \(y = mx + c\), where \(m\) and \(c\) are 32-bit constants.
              \item \(\texttt{main.c}\): Calls \(\texttt{linear()}\) with different values of \(x\).
          \end{itemize}
    \item Fill in the following table.
          Specify the type to be one of:
          function, read-only data, initialized data, uninitialized data, local data.
\end{enumerate}

\subsection*{Solution}

\begin{table}[htbp]
    \centering
    \begin{tabular}{|l|l|l|l|}
	\hline
	\textbf{\large symbol} & \textbf{\large type} & \textbf{\large section}      & \textbf{\large address} \\
	\hline
	\(\texttt{main}\)      & function             & \verb|.text.main|            & \hex{0x00000065}        \\
	\hline
	\(\texttt{i}\)         & local data           & \verb|.bss.stack_memory|     & \hex{0x2000001C}        \\
	\hline
	\(\texttt{linear}\)    & function             & \verb|.text.linear|          & \hex{0x00000055}        \\
	\hline
	\(\texttt{x\_array}\)  & initialised data*    & \verb|.data.x_array|         & \hex{0x20000004}        \\
	\hline
	\(\texttt{y\_array}\)  & uninitialised data   & \verb|.bss.y_array|          & \hex{0x20000094}        \\
	\hline
	\(\texttt{m}\)         & read-only data       & \verb|.rodata.m|             & \hex{0x000000AC}        \\
	\hline
	\(\texttt{c}\)         & read-only data       & \verb|.rodata.c|             & \hex{0x000000A8}        \\
	\hline
	(vector table)         & read-only data       & \verb|.rodata.__Vectors|     & \hex{Ox00000000}        \\
	\hline
	(reset handler)        & function             & \verb|.text.Reset_Handler|   & \hex{0x00000045}        \\
	\hline
	(default handler)      & function             & \verb|.text.Default_Handler| & \hex{0x00000041}        \\
	\hline
\end{tabular}

\end{table}

The corresponding symbols window is shown in Figure~\ref{fig:symbols}.

\newpage
As for the local variable \( \texttt{i} \), it is stored in the stack as and when required, on run-time.
It turns out that the Stack Pointer (SP) points to \( \hex{0x2000001C} \) when the program was running, which is where the variable \( \texttt{i} \) is stored.

The \( \texttt{.bss} \) section indicates ``block starting symbol'', indicating that the variables are declared but not being assigned a value yet.

\vspace*{3em}
\begin{figure*}[h]
    \centering
    \includegraphics[width=0.8\textwidth]{figures/images/symbols.png}
    \caption{
        Symbols window
    }\label{fig:symbols}
\end{figure*}
