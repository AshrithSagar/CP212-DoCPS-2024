\section*{Question 4}

What determines the period for which a row should remain on?

\subsection*{Solution}

The period for which a row should remain on is determined by the \underline{refresh rate} of the display, which is the number of times the display is updated per second.

Displays work on the principle of persistence of vision, by noticing the fact that the human eye retains an image for a fraction of a second after the source has been removed.
Typically, this persistence duration is around 1/16th of a second or about 60 milliseconds.
This phenomenon is exploited to show a complete image by refreshing the display at a rate faster than the persistence duration, so that the image appears continuous to the human eye.
For flicker-free display, the refresh rate should be at least 60 Hz.

Further, we have the following
\[
    t_{ON} \text{ for each row}
    =
    \frac{1}{\text{Refresh rate} \times \text{Number of rows}}
\]

For a refresh rate of, say 60 Hz, and a display with 5 rows, we have
\[
    t_{ON} = \frac{1}{60 \times 5} = 3.33 \text{ ms}
\]
i.e., each row should be illuminated atmost for 3.33 ms to not observe any flickers.
