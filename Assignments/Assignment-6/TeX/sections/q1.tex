\section*{Question 1}

How many GPIO pins does NRF52833 have?
\begin{enumerate}[label= (\alph*)]
    \item The schematic tells us that the part number is nRF52833-QIAA.\@
    \item Section 10.3 tells us the meaning of QIAA.\@
    \item Section 7.1.1 tells us pin configuration for QFN73 package.
\end{enumerate}

\subsection*{Solution}

The general purpose input/output pins (GPIOs) are grouped into two ports, P0 and P1, with each port having up to 32 GPIOs.\\
For \texttt{nRF52833}, there are 32+10=\underline{42 GPIO pins}.

Port P0 has 32 GPIO pins, numbered from 0 to 31.

Port P1 has 10 GPIO pins, numbered from 0 to 9.

\subsubsection*{Registers}

The GPIO port peripheral implements up to 32 pins, \texttt{PIN0} through \texttt{PIN31}.
Each of these pins can be individually configured in the \texttt{PIN\_CNF[n]} registers (n=0..31).

\begin{itemize}
    \item \texttt{OUT} --- Write GPIO port
    \item \texttt{OUTSET} --- Set individual bits in GPIO port
    \item \texttt{OUTCLR} --- Clear individual bits in GPIO port
    \item \texttt{IN} --- Read GPIO port
    \item \texttt{DIR} --- Direction of GPIO pins
    \item \texttt{DIRSET} --- Set bits in DIR register
    \item \texttt{DIRCLR} --- Clear bits in DIR register
    \item \texttt{LATCH} --- Latch register
    \item \texttt{DETECTMODE} --- Select between DETECT or LDETECT signal mode
    \item \texttt{PIN\_CNF[n]} --- Configuration of GPIO pins
          \subitem~n = 0 to 31
\end{itemize}

\clearpage
\begin{figure}[htbp]
    \centering
    \includegraphics[page=1, width=\textwidth]{figures/q2/_}
    \vspace*{-3.5em}
    \caption{
        Order code abbreviations and their meaning for \texttt{nRF52833} series
    }\label{fig:q1-order}
\end{figure}

The manual gives the order codes and definitions for the \texttt{nRF52833} series, as shown in Figure~\ref{fig:q1-order} above.

\begin{figure}[htbp]
    \centering
    \includegraphics[width=0.65\textwidth]{figures/images/q1-ball.png}
    \caption{
        Pin configuration for \texttt{QFN73} package.
        Consists of 73 pins/balls.
    }\label{fig:q1-ball}
\end{figure}

Pin configuration for \texttt{QFN73} package is shown in Figure~\ref{fig:q1-ball} above.
We can verify that there are 73 pins here.
In the table given in the manual for this package, we can see that there are 42 GPIO pins out of the 73 available ones.
