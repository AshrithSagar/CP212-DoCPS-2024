\section*{Question 5}

What does FPU do?

\subsection*{Solution}

A floating-point unit (FPU), also called math co-processor, is a hardware unit dedicated to perform floating point calculations effectively and efficiently.
Traditionally, floating point operations were carried out using libraries, i.e., in software.
An FPU tries to offload these operations from the CPU to improve performance, and does it in hardware.
FPUs support floating point addition, subtraction, multiplication, division, and other operations.
Now-a-days, FPUs are usually integrated into the CPU itself, and are not separate units.

When a CPU is executing a program and requires a floating point operation to be carried out, there are three possible ways to perform it:
\begin{enumerate}[itemsep=3pt,parsep=0pt,topsep=5pt,partopsep=0pt]
    \item In software, by using a floating point emulator / library.
    \item In hardware, using add-on FPUs.
    \item In hardware, using integrated FPUs.
\end{enumerate}

FPUs are quite fast and efficient compared to the software alternatives.
One limitation of FPUs is that they cannot perform arbitrary-precision arithmetic on their own, i.e., they only support a finite number of operations due to hardware limitations.
Using libraries, it is possible to perform arbitrary-precision arithmetic, if need be.
