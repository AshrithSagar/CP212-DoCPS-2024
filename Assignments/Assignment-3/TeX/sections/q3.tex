\section*{Question 3}

Set a variable \verb|float x = 1.0|.
View it as an unsigned long in the memory window.
What value do you see?
Explain it theoretically.

\subsection*{Solution}

Floating-point numbers are stored in the memory according to the IEEE 754 standard.
According to it, the variable \texttt{float x = 1.0} is stored internally as shown below in Figure~\ref{fig:q3}.

\begin{figure*}[htb]
    \centering
    \includegraphics[width=0.95\linewidth]{figures/q3/_}
    \caption{
        Memory representation of \texttt{float x = 1.0} according to the IEEE 754 standard.
    }\label{fig:q3}
\end{figure*}

For 32-bit processors, the standard specifies the following format for single-precision floating-point numbers, with the MSB bit for the sign, the next 8 bits for the exponent and the next 23 bits for the mantissa.
The sign bit is 0 for positive numbers and 1 for negative numbers.
The exponent is stored in excess-127 notation, i.e., the actual exponent is the stored value minus 127.
The mantissa is stored in normalized form.

For \texttt{float x = 1.0}, the sign bit is 0 and the exponent is 127.

Therefore, the variable \texttt{float x = 1.0} is stored as \(\boxed{\texttt{0x3F800000}}\) in memory.
