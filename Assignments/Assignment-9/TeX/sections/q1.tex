\section*{Question 1}

What does ``A'' in UART stand for?
What does it mean?
What are its advantages and disadvantages?

\subsection*{Solution}

UART stands for ``Universal Asynchronous Receiver/Transmitter'', thereby the ``A'' stands for ``Asynchronous''.
Asynchronous communication means that data is sent without a shared clock signal between the sender and receiver.
Instead, the devices agree on a specific baud rate (data transmission speed) and use start and stop bits to indicate the beginning and end of each data byte.
This allows for flexible communication without needing a constant clock signal.

\subsubsection*{Advantages of UART}
\begin{itemize}[noitemsep]
    \item Simple to implement.
          Requires only two wires for communication, \texttt{TX} and \texttt{RX}.
    \item Can be used to communicate between devices with different clock speeds.
    \item Can be used to communicate over long distances, since synchronizing clocks over long distances can be challenging.
\end{itemize}

\subsubsection*{Disadvantages of UART}
\begin{itemize}[noitemsep]
    \item Requires additional bits for synchronization which has an overhead. \\
          This reduces the data throughput.
    \item If the baud rates are not perfectly matched, data corruption can occur. \\
          Timing discrepancies can lead to misinterpretation of the data.
\end{itemize}
