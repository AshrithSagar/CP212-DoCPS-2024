\section*{Question 2}

How should you compute the refresh rate?

\subsection*{Solution}

The refresh rate of the display is the number of times the display is updated per second.
To compute the refresh rate, we can do it as follows:
\begin{align*}
     &
    (\text{Time for one update cycle of the display})
    =
    (t_{ON} \text{ for each row})
    \times
    (\text{Number of rows})
    \\ &
    \implies
    \text{Refresh rate}
    =
    \frac{1}{
        (\text{Time for one update cycle of the display})
    }
    \\ &
    \therefore
    \boxed{
        \text{Refresh rate}
        =
        \frac{1}{
            (t_{ON} \text{ for each row})
            \times
            (\text{Number of rows})
        }
    }
\end{align*}
Say, for example, with \( (\text{Number of rows}) = 5 \) and \( (t_{ON} \text{ for each row}) = 3 \text{ ms} \), we have
\begin{align*}
    \implies
    \text{Refresh rate}
     & =
    \frac{1}{3 \text{ ms} \times 5}
    =
    \frac{1}{15 \text{ ms}}
    =
    \frac{1000}{15} \text{Hz}
    =
    66.67 \text{ Hz}
\end{align*}
Thus, the refresh rate of the display is approximately \( 66.67 \text{ Hz} \).
