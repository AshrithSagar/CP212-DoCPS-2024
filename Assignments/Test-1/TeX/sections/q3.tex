\section*{Question 3}

To understand about hard and soft floating-point libraries, I then changed the ``Floating Point Hardware'' option ``Target'' tab from ``Single Precision'' to ``Not used''.

I saw that arithmetic operations in the disassembled code changed to:
\vspace*{1em}

\begin{tabular}{llll}
  \textbf{Address} & \textbf{Code}  & \textbf{Disassembly} &                                         \\
  \hex{0x000004F6} & \hex{2100}     & \texttt{MOVS}        & \texttt{r1, \#0x00}                     \\
  \hex{0x000004F8} & \hex{F2C401A0} & \texttt{MOVT}        & \texttt{r1, \#0x40A0}                   \\
  \hex{0x000004FC} & \hex{F000F878} & \texttt{BL.W}        & \texttt{0x000005F0} \verb|__aeabi_fmul| \\
  \hex{0x00000500} & \hex{F04F4180} & \texttt{MOV}         & \texttt{r1, \#0x40000000}               \\
  \hex{0x00000504} & \hex{F000F812} & \texttt{BL.W}        & \texttt{0x0000052C} \verb|__aeabi_fadd| \\
\end{tabular}
\vspace*{1em}

What do you think these operations do?

\subsection*{Solution}

When we disable the FPU, the processor performs floating point operations using software libraries instead.
As seen from the disassembly window, we can see that the processor tries to call the respective floating point multiplication function \verb|__aeabi_fmul| and floating point addition function \verb|__aeabi_fadd|.
This computation may take a few clock cycles to complete, before the processor resumes to execute the following instructions.

We can see that the floating point functions are being branched to using the \texttt{BL.W} instruction, which is a branch with link instruction.
