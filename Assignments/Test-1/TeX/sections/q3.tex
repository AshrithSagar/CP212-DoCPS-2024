\section*{Question 3}

To understand about hard and soft floating-point libraries, I then changed the ``Floating Point Hardware'' option ``Target'' tab from ``Single Precision'' to ``Not used''.

I saw that arithmetic operations in the disassembled code changed to:
\vspace*{0.5em}

\begin{tabular}{llll}
  \textbf{Address} & \textbf{Code}  & \textbf{Disassembly} &                                         \\
  \hex{0x000004F6} & \hex{2100}     & \texttt{MOVS}        & \texttt{r1, \#0x00}                     \\
  \hex{0x000004F8} & \hex{F2C401A0} & \texttt{MOVT}        & \texttt{r1, \#0x40A0}                   \\
  \hex{0x000004FC} & \hex{F000F878} & \texttt{BL.W}        & \texttt{0x000005F0} \verb|__aeabi_fmul| \\
  \hex{0x00000500} & \hex{F04F4180} & \texttt{MOV}         & \texttt{r1, \#0x40000000}               \\
  \hex{0x00000504} & \hex{F000F812} & \texttt{BL.W}        & \texttt{0x0000052C} \verb|__aeabi_fadd| \\
\end{tabular}
\vspace*{0.5em}

What do you think these operations do?

\vspace*{-2em}
\subsection*{Solution}

The processor performs \underline{floating point operations using software libraries} instead when the FPU is disabled.
As seen from the disassembly window, we can see that the processor tries to call the respective floating point multiplication routine \verb|__aeabi_fmul| and floating point addition routine \verb|__aeabi_fadd|.
This computation may take a few clock cycles to complete, before the processor resumes to execute the following instructions.

We can see that the floating point routines are being branched to using the \texttt{BL.W} instruction, which is the branch with link instruction.
These routines are provided in \( \texttt{fplib}\), which is the software floating-point library.
They are a part of the ARM EABI (Embedded Application Binary Interface).

\vspace*{-1em}

\begin{figure*}[htbp]
  \centering
  \includegraphics[page=1, width=0.9\textwidth]{figures/q3/_}
  \setlength{\abovecaptionskip}{-12pt}
  \caption{
    The number \( 5 \) in the IEEE 754 format is \( \hex{0x40A00000} \).
  }
\end{figure*}

\begin{figure*}[htbp]
  \centering
  \includegraphics[page=2, width=0.9\textwidth]{figures/q3/_}
  \setlength{\abovecaptionskip}{-12pt}
  \caption{
    The number \( 2 \) in the IEEE 754 format is \( \hex{0x40000000} \).
  }
\end{figure*}

\clearpage
\subsubsection*{Instruction breakdown}

\begin{tabular}{llll}
  \hex{0x000004F6} & \hex{2100} & \texttt{MOVS} & \texttt{r1, \#0x00} \\
\end{tabular}

Moves the immediate value \( 0 \) into register \( \texttt{R1} \).
The \texttt{S} suffix indicates to sets the flags based on the result.

\vspace*{1em}

\begin{tabular}{llll}
  \hex{0x000004F8} & \hex{F2C401A0} & \texttt{MOVT} & \texttt{r1, \#0x40A0} \\
\end{tabular}

Moves the immediate value \( \hex{0x40A0} \) into the upper half of register \( \texttt{R1} \).
\( \texttt{MOVT} \) indicates Move Top, and writes an immediate value to the top halfword of the register while the bottom halfword remains unchanged.
Thereby, the register \( \texttt{R1} \) now contains the value \( \hex{0x40A00000} \).
This can be seen to be the floating point representation of the number \( 5 \) in IEEE 754 format.

\vspace*{1em}

\begin{tabular}{llll}
  \hex{0x000004FC} & \hex{F000F878} & \texttt{BL.W} & \texttt{0x000005F0} \verb|__aeabi_fmul| \\
\end{tabular}

Branch with Link (Immediate) calls the subroutine \verb|__aeabi_fmul| relative to the PC-relative address.

\vspace*{1em}

\begin{tabular}{llll}
  \hex{0x00000500} & \hex{F04F4180} & \texttt{MOV} & \texttt{r1, \#0x40000000} \\
\end{tabular}

Moves the immediate value \( \hex{0x40000000} \) into register \( \texttt{R1} \), which is the number floating point representation of the number \( 2 \) in IEEE 754 format.

\vspace*{1em}

\begin{tabular}{llll}
  \hex{0x00000504} & \hex{F000F812} & \texttt{BL.W} & \texttt{0x0000052C} \verb|__aeabi_fadd| \\
\end{tabular}

Branch with Link (Immediate) calls the subroutine \verb|__aeabi_fadd| relative to the PC-relative address.
