\section*{Question 2}

I ``upgraded'' the code to floating point operations, and modified \texttt{linear.c} as shown below.
\vspace*{-2em}
\begin{lstlisting}[language=C, frame=single, caption={linear.c}]
float linear(float x)
{
  float y;
  y = 5 * x + 2;

  return y;
}
\end{lstlisting}

The disassembled code looked like this:
\vspace*{1em}

\begin{tabular}{llll}
  \textbf{Address} & \textbf{Code}  & \textbf{Disassembly} &                     \\
  \hex{0x00000500} & \hex{EEB11A04} & \texttt{VMOV.F32}    & \texttt{s2, \#5}    \\
  \hex{0x00000504} & \hex{EE200A01} & \texttt{VMUL.F32}    & \texttt{s0, s0, s2} \\
  \hex{0x00000508} & \hex{EEB01A00} & \texttt{VMOV.F32}    & \texttt{s2, \#2}    \\
  \hex{0x0000050C} & \hex{EE300A01} & \texttt{VADD.F32}    & \texttt{s0, s0, s2} \\
  \hex{0x00000510} & \hex{4770}     & \texttt{BX}          & \texttt{lr}         \\
\end{tabular}
\vspace*{1em}

Briefly explain what these instructions do.

\vspace*{-1em}
\subsection*{Solution}

Changing the code to floating point operations converts the instructions to vector operations when the Floating Point Unit (FPU) is enabled.
This can be seen by the \( \texttt{V} \) prefix for each of the instructions in the disassembly.
The FPU does the floating-point arithmetic using hardware.
The routine calculates the result \( (5.0 \times x + 2.0) \) using floating-point operations.
\vspace*{0.5em}

\begin{tabular}{llll}
  \hex{0x00000500} & \hex{EEB11A04} & \texttt{VMOV.F32} & \texttt{s2, \#5} \\
\end{tabular}

This places an immediate value of \( 5.0 \) into the single-precision floating-point register \( \texttt{S2} \).
\vspace*{0.01em}

\begin{tabular}{llll}
  \hex{0x00000504} & \hex{EE200A01} & \texttt{VMUL.F32} & \texttt{s0, s0, s2} \\
\end{tabular}

This operation multiplies the values in the single-precision floating-point registers \( \texttt{S0} \) and \( \texttt{S2} \) and stores the result in \( \texttt{S0} \).
\vspace*{1em}

\begin{tabular}{llll}
  \hex{0x00000508} & \hex{EEB01A00} & \texttt{VMOV.F32} & \texttt{s2, \#2} \\
\end{tabular}

This places an immediate value of \( 2.0 \) into the single-precision floating-point register \( \texttt{S2} \).
\vspace*{0.01em}

\begin{tabular}{llll}
  \hex{0x0000050C} & \hex{EE300A01} & \texttt{VADD.F32} & \texttt{s0, s0, s2} \\
\end{tabular}

This performs a floating-point addition of the values in the single-precision floating-point registers \( \texttt{S0} \) and \( \texttt{S2} \) and stores the result in \( \texttt{S0} \).
\vspace*{1em}

\begin{tabular}{llll}
  \hex{0x00000510} & \hex{4770} & \texttt{BX} & \texttt{lr} \\
\end{tabular}

This instruction branches to the address stored in the link register, performing a return to the function call.
