\section*{Question 2}

I ``upgraded'' the code to floating point operations, and modified \texttt{linear.c} as shown below.
\begin{lstlisting}[language=C, frame=single, caption={linear.c}]
float linear(float x)
{
  float y;
  y = 5 * x + 2;

  return y;
}
\end{lstlisting}

The disassembled code looked like this:
\vspace*{1em}

\begin{tabular}{llll}
  \textbf{Address} & \textbf{Code}  & \textbf{Disassembly} &                     \\
  \hex{0x00000500} & \hex{EEB11A04} & \texttt{VMOV.F32}    & \texttt{s2, \#5}    \\
  \hex{0x00000504} & \hex{EE200A01} & \texttt{VMUL.F32}    & \texttt{s0, s0, s2} \\
  \hex{0x00000508} & \hex{EEB01A00} & \texttt{VMOV.F32}    & \texttt{s2, \#2}    \\
  \hex{0x0000050C} & \hex{EE300A01} & \texttt{VADD.F32}    & \texttt{s0, s0, s2} \\
  \hex{0x00000510} & \hex{4770}     & \texttt{BX}          & \texttt{lr}         \\
\end{tabular}
\vspace*{1em}

Briefly explain what these instructions do.

\subsection*{Solution}
