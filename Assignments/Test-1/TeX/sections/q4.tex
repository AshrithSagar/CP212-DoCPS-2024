\section*{Question 4}

I continued running the code.
However, while running the code, I forgot to update the prototype in \texttt{main.c}, which remained:
\begin{lstlisting}[language=C]
int linear(int x);
\end{lstlisting}

I stepped through the code from \texttt{main} onwards.

In \texttt{main}, the value of global variable \texttt{x} was correctly shown as \texttt{10}.
But when I stepped into \texttt{linear}, I saw the argument \texttt{x} to have the value \texttt{1.40129846e-44}.
And when the function returned, the value of global variable \texttt{y} was shown as \texttt{1073741824}.

Why is this happening?
Is the compiler buggy?

\subsection*{Solution}

\begin{figure*}[htbp]
    \centering
    \includegraphics[width=0.9\textwidth]{figures/q4/_}
    \setlength{\abovecaptionskip}{-12pt}
    \caption{
        The value \( \hex{0xA} \) interpreted as in the IEEE 754 format
    }\label{fig:q4}
\end{figure*}

The issue caused by the type mismatch causes the compiler to interpret the value of \( x \) as a floating point number.
Initially, the value of \( x \) is correctly interpreted as \( 10 \), as an integer.
Since the prototype specifies the type to be an integer, the compiler doesn't perform any type conversion when it is passed to the function \texttt{linear}.
Thereby the contents of the register are as shown in Figure~\ref{fig:q4}.
However, the function \texttt{linear} expects a floating point number as per what was written the function body, thereby causing the compiler to interpret the value of \( x \) as a floating point number.
This is \underline{not the compiler's fault}, but rather just a programming error.

To check what this number is actually, we can compute it as follows:
When the exponent is \( \hex{00000000} \), a special rule is followed, where the value is treated as a \textit{denormalised number}, thereby the leading 1 in the mantissa is not considered and the actual exponent is taken as \( -126 \), i.e., the value here is given by:
\[
    \implies
    + 2^{ -126 } \times 0.\underbrace{0000000000000000000}_{19\text{ zeros}}1010_{2}
\]
\begin{equation*}
    \implies
    2^{ - 126 } \times (2^{-20} + 2^{-22})
    =
    1.4012984643 \times 10^{-44}
\end{equation*}

Now, when the function \texttt{linear} is run, it returns the value \( 5x + 2 \), which is computed in floating point arithmetic by first having \( 5 \) and \( 2 \) represented as floating point numbers, as seen previously in Figure~\ref{fig:q3-5} and Figure~\ref{fig:q3-2}, respectively.

Multiplying the mantissa's, we get
\begin{align*}
    \implies
    (2^{-20} + 2^{-22}) \times (1 + 2^{-2})
     & =
    2^{-20} + 2^{-22} + 2^{-22} + 2^{-24}
    \\ & =
    (2^{-20} + 2^{-21}+ 2^{-24})
    \\ & =
    2^{-20} \times (1 + 2^{-1}+ 2^{-4})
\end{align*}
Adding the exponents, we get
\[
    \implies
    (-126) + (129 - 127)
    =
    (3 - 127)
\]
Normalising the mantissa, we get
\begin{align*}
    \implies
     &
    + 2^{3 - 127} \times 2^{-20} \times (1 + 2^{-1}+ 2^{-4})
    \\ & =
    + 2^{-17 - 127} \times (1 + 2^{-1}+ 2^{-4})
\end{align*}
This is the value of \( 5x \).
Now, addition with \( 2 \) can be done by
\begin{align*}
     &
    \Big( + 2^{-17 - 127} \times (1 + 2^{-1}+ 2^{-4}) \Big)
    +
    \Big( + 2^{128 - 127} \times (1) \Big)
    \\ & =
    2^{-127} \times (2^{-17} + 2^{-18} + 2^{-21} + 2^{128})
    \\ & =
    2^{128 - 127} \times (1 + 2^{-145} + 2^{-146} + 2^{-149})
    \\ & \implies
    + 2^{128 - 127} \times (1)
\end{align*}
Thereby, the value remains as 2, since the other term is too small.
