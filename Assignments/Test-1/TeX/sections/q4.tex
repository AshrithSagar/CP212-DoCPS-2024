\section*{Question 4}

I continued running the code.
However, while running the code, I forgot to update the prototype in \texttt{main.c}, which remained:
\begin{lstlisting}[language=C]
int linear(int x);
\end{lstlisting}

I stepped through the code from \texttt{main} onwards.

In \texttt{main}, the value of global variable \texttt{x} was correctly shown as \texttt{10}.
But when I stepped into \texttt{linear}, I saw the argument \texttt{x} to have the value \texttt{1.40129846e-44}.
And when the function returned, the value of global variable \texttt{y} was shown as \texttt{1073741824}.

Why is this happening?
Is the compiler buggy?

\subsection*{Solution}

\begin{figure*}[htbp]
    \centering
    \includegraphics[width=0.9\textwidth]{figures/q4/_}
    \setlength{\abovecaptionskip}{-12pt}
    \caption{
        The value \( \hex{0xA} \) interpreted as in the IEEE 754 format
    }\label{fig:q4}
\end{figure*}

The issue arises from type mismatch in the function prototype.
Initially, before passing to the function \( \texttt{linear} \), the value of \( x \) is correctly interpreted as \( 10 \), i.e., as an integer.
Thereby the contents of the register are \( \hex{0xA} \).
However, the function \texttt{linear} expects a floating point number as per what was written the function body, thereby causing the compiler to interpret the value of \( x \) as a floating point number, as shown in Figure~\ref{fig:q4}.
This is \underline{not the compiler's fault}, but rather just a programming error.

To check what this number is actually, we can compute it as follows:
When the exponent is \( \hex{00000000} \), a special rule is followed, where the value is treated as a \textit{denormalised number} /\textit{subnormal number}, thereby the hidden leading 1 in the mantissa is not considered and the actual exponent is taken as \( -126 \), i.e., the value here is given by:
\[
    \implies
    {(-1)}^0 \times 2^{ -126 } \times 0.\underbrace{0000000000000000000}_{19\text{ zeros}}1010_{2}
\]
\begin{equation*}
    \implies
    2^{ - 126 } \times (2^{-20} + 2^{-22})
    =
    \boxed{
        1.4012984643 \times 10^{-44}
    }
\end{equation*}

Thereby, the integer 10 is treated as a very small floating-point number \( 1.4012984643 \times 10^{-44} \) and the function \texttt{linear} computes the value \( 5x + 2 \) based on this value.
The value of \( 5x \) is computed as \( 5 \times 1.4012984643 \times 10^{-44} \), which is effectively zero, due to limitations of the floating point representation.

We can verify the computations  manually.
Multiplying the mantissa's, we get
\begin{align*}
    \implies
    (2^{-20} + 2^{-22}) \times (1 + 2^{-2})
     & =
    2^{-20} + 2^{-22} + 2^{-22} + 2^{-24}
    \\ & =
    2^{-20} + 2^{-21}+ 2^{-24}
    \\ & =
    2^{-20} \times (1 + 2^{-1}+ 2^{-4})
\end{align*}
Adding the exponents, we get
\[
    \implies
    (-126) + (129 - 127)
    =
    (3 - 127)
\]
Normalising the mantissa, we get
\begin{align*}
    \implies
     &
    + 2^{3 - 127} \times 2^{-20} \times (1 + 2^{-1}+ 2^{-4})
    \\ & =
    + 2^{-17 - 127} \times (1 + 2^{-1}+ 2^{-4})
\end{align*}
But this value can't be represented in the conventional floating point representation, since the exponent (-17) is too small.
Hence, this value becomes zero, which is the value of \( 5x \) in floating point representation.

When the function computes \( 5x + 2\), the result is dominated by the constant 2.
Therefore, the result of \( 5x + 2\) yields 2.0, i.e., \( \hex{0x40000000} \) in the IEEE 754 format, which is the value that is stored at the end of the function call.
But since the prototype defines the return type as \( \texttt{int} \), the value is interpreted as an integer.
The decimal representation of \( \hex{0x40000000} \) is \( 2^{30} = \boxed{1073741824} \), which is the value that is shown in the global variable \( y \).

The \underline{compiler is not buggy}, but rather it is the fault of the programmer here.
