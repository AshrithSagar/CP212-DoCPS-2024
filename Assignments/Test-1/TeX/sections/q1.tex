\section*{Question 1}

I wrote the following code in two files:
\begin{lstlisting}[language=C, frame=single, caption={main.c}]
int linear(int x);

int x = 10;
int y;

int main(void) {
  y = linear(x);

  return 0;
}
\end{lstlisting}
\begin{lstlisting}[language=C, frame=single, caption={linear.c}]
int linear(int x)
{
  int y;
  y = 5 * x + 2;

  return y;
}
\end{lstlisting}

I compiled the code for Cortex-M4 processor and loaded on the simulator.\\
The disassembly code looked like this:
\vspace*{1em}

\begin{tabular}{llll}
  \textbf{Address} & \textbf{Code}  & \textbf{Disassembly} &                              \\
  \hex{0x00000500} & \hex{EB000080} & \texttt{ADD}         & \texttt{r0, r0, r0, LSL \#2} \\
  \hex{0x00000504} & \hex{3002}     & \texttt{ADDS}        & \texttt{r0, r0, \#0x02}      \\
  \hex{0x00000506} & \hex{4770}     & \texttt{BX}          & \texttt{lr}                  \\
\end{tabular}
\vspace*{1em}

Answer the following:
\begin{enumerate}[itemsep=0mm]
  \item Explain briefly what each instruction does.
  \item Explain the instruction encoding of the arithmetic instructions.
\end{enumerate}

\subsection*{Solution}

\subsubsection*{Instruction breakdown}

\begin{tabular}{llll}
  \hex{0x00000500} & \hex{EB000080} & \texttt{ADD} & \texttt{r0, r0, r0, LSL \#2} \\
\end{tabular}
\vspace*{1em}

This instruction effectively \underline{multiplies the number present in register \( \texttt{R0} \) by 5}, and stores it in \( \texttt{R0} \).
A left shifted version of \( \texttt{R0} \) by 2 bits is added to \( \texttt{R0} \), which effectively computes \( 4 \times \texttt{R0} \) and adds this to \( \texttt{R0} \), thereby producing \( 5 \times \texttt{R0} \) at the end.
\vspace*{1em}

\begin{tabular}{llll}
  \hex{0x00000504} & \hex{3002} & \texttt{ADDS} & \texttt{r0, r0, \#0x02} \\
\end{tabular}
\vspace*{1em}

This instruction \underline{adds the number 2} to the number present in \( \texttt{R0} \), and stores the result back in \( \texttt{R0} \), and updates the flags accordingly.

\vspace*{1em}

\begin{tabular}{llll}
  \hex{0x00000506} & \hex{4770} & \texttt{BX} & \texttt{lr} \\
\end{tabular}
\vspace*{1em}

This instruction branches to the address present in the link register, which acts as a \underline{return to the function call}.
The function call \( \texttt{linear(.)} \) accepts one input in \( \texttt{R0} \) and returns \( (5 \times \texttt{R0} + 2) \) back in \( \texttt{R0} \), as required by the AAPCS specification.

\subsubsection*{Instruction encoding}
